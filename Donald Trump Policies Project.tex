\documentclass[12pt,letterpaper]{article}

\usepackage[margin=1.5in]{geometry}
\usepackage[english]{babel}
\usepackage[utf8x]{inputenc}
\usepackage{amsmath}
\usepackage{amssymb} 
\usepackage[retainorgcmds]{IEEEtrantools}
\usepackage{graphicx}
\usepackage{tabularx}
\usepackage{subfig}
\usepackage{kpfonts}    % for nice fonts
\usepackage{microtype} 
\usepackage{booktabs}   % for nice tables
\usepackage{bm}         % for bold math
\usepackage{listings}   % for inserting code
\documentclass[12pt,letterpaper]{article}

\usepackage[margin=1.5in]{geometry}
\usepackage[english]{babel}
\usepackage[utf8x]{inputenc}
\usepackage{amsmath}
\usepackage{amssymb} 
\usepackage[retainorgcmds]{IEEEtrantools}
\usepackage{graphicx}
\usepackage{tabularx}
\usepackage{subfig}
\usepackage{kpfonts}    % for nice fonts
\usepackage{microtype} 
\usepackage{booktabs}   % for nice tables
\usepackage{bm}         % for bold math
\usepackage{listings}   % for inserting code
\usepackage{verbatim}   % useful for program listings
\usepackage{color}  
\usepackage[colorlinks=true]{hyperref}
\usepackage[colorinlistoftodos]{todonotes}
\usepackage{natbib}
\usepackage{setspace}
\usepackage{indentfirst}

\doublespacing
\setlength {\marginparwidth }{2cm}
\begin{document}

%+Title
\title{\textbf{Donald Trump: The Removal of Income Tax}}
\author{George Johnson\footnote{Economics B.S University Houston. Email: Gtjohnso@CougarNet.Uh.EDU} \& Aidan Brogan\footnote{Economics B.S University Houston. Email: ApbroganCougarNet.UH.EDU}}
\date{December 2, 2024}
\maketitle

\textbf{Abstract:} This paper explores the potential economic impact of Donald Trump's proposed elimination of income tax on the United States (U.S.) economy. Building upon thoughts from experts and previous studies, we adapt and expand these frameworks to model and analyze the full scope of this tax policy. Our simulation examines the effects of the economic indicators of GDP and wealth distribution. Based on the outcomes, we discuss the potential future policy measures. These recommendations aim to facilitate a smoother implementation of the proposed plan, should it secure legislative approval.

\newpage
\section{Introduction}

 \indent Donald Trump's victory in the 2024 U.S. presidential election has brought attention to his proposal to eliminate federal income tax. Such a plan would have significant effects on the U.S. economy: experts argue that this proposal is fundamentally and mathematically unfeasible \citep{Max_Zahn}. Federal income tax accounts for a substantial portion of government revenue. It also funds critical programs such as Social Security, Medicare, and national defense. The removal of this revenue stream could drastically alter the fiscal landscape, leaving significant gaps in funding for essential services.
 
\subsection*{Motivation}

 \indent This shift would also affect economic functions, potentially increasing income inequality and altering market behavior. There, however, are strategies policymakers could explore to mitigate risks. Understanding the economic implications of this proposal is essential as lawmakers consider its feasibility. Balancing innovation in tax policy with financial sustainability will determine the success or failure of such a trans-formative plan.

 
\section{Background}

\indent \cite{Vermeer_Timothy} goes into detail on the impact of individual income tax and its changes on Economic growth, making findings related to GDP, upward job mobility, and private investment. They find that higher income tax rates lead to lower GDP and private investment as they have a negative relationship. Higher rates also make it harder for employed people to find a better paying job within a year.  

\indent Other research papers study how no income taxes make income inequality worse. \cite{Hellmann_Melissa}, states that when you remove the income tax and rely on sales taxes most of the financial burden falls on the lower and middle income families. The article shows the differences between Oregon which has a relatively progressive income tax vs Washington which has no income tax and relies on sales taxes. This study showed that lower income families in Washington payed around 18 percent of their annual income to state/local taxes while the wealthier brackets payed only around 3 percent showing the wealth discrepancy brought about by removing income tax.

\indent \cite{Steindel_Charles} studies the effect tax changes have on consumer spending. The data from the 1975 tax rebate and 1982 tax cut show that changes in income tax do lead to lower personal savings rates. However, these reactions only occur once consumers realize a difference in their take-home pay. It also suggests that reactions are stronger with permanent tax changes as opposed to temporary ones. This is shown in the initial lowering of the personal savings rate post 1968 tax surcharge, which eventually raised back to higher levels once the surcharge was lowered to 5 percent and given an expiration date.

\indent \cite{Cornforth_Ed}, shows the effects other forms of taxation have on the economy in comparison to income tax. The article studies both indirect taxes (such as Value-added tax) and corporate taxes in comparison to income tax and the changes they make in relation to real GDP and potential output. The data represents that income tax has least affect on real GDP only slightly lowering it initially and that an income tax actually raises the potential output of the economy by 0.07 to 0.08 percent in the long run. The most damaging tax to real GDP and potential output was a corporate tax showing a -0.23 percent change in the long-run GDP and an exponential drop in potential output reaching -0.35 to -0.37 percent change. The indirect tax numbers were in between the two other forms with a slightly lower real GDP change than income tax and no real difference in potential output. These percents appear small when dealing with trillions of dollars however, they make a large difference.

\indent The studies of \cite{Goss_Jacob} look into the impacts of various taxes on a state level. Their research shows that an income tax is the only form of tax that affects personal income growth, while corporate tax and sales taxes do not. However, income tax is the most stable tax while corporate and sales taxes are more volatile. The study concludes that you should not rely on strictly corporate and sales taxes like some states do if you want to have a steady and nonvolatile tax base.

\indent Lastly, we see the affect income tax cuts might have on national debt \citep{Kogan_Bobby}. This study takes data from both Bush's tax cuts and Trump's first-term tax-cuts where income taxes were reduced in both instances. These tax cuts have led to about a 10 trillion dollars debt increase since the year of 2001 only increasing the debt ratio. This study also compares the tax rates of the U.S and other countries with the U.S being considered a low-tax country. Tax rates in the U.S are lower than both the European and OCED average Total revenues as a 
percentage of GDP.

\indent From past research we can see that income tax has a high effect on the individual. In our analysis we will analyze this effect by calculating the overall new GDP then using this to find GDP per capita. We will also see how wealth disparity changes to represent who would be most impacted, the wealthy or the lower class by these policy changes.

\section{Data and Equations}

\indent In order to calculate the effects of the removal of income tax and increase in tariffs at Trumps desired plan of 100\%, we will modify the GDP equation to find the new GDP per capita and the wealth disparity equation.

\subsection{Equations}

\subsection*{Change in GDP (\( \Delta \text{GDP} \))}

\[
\Delta \text{GDP} = (\beta_C + \beta_I + \beta_G) \cdot (\Delta T + \Delta \text{Tariff}) \cdot \text{GDP}_{\text{current}}
\]

\subsection*{New GDP (\( \text{GDP}_{\text{new}} \))}

\[
\text{GDP}_{\text{new}} = \text{GDP}_{\text{current}} + \Delta \text{GDP}
\]

\subsection*{GDP Per Capita (\( \text{GDP}_{\text{pc,new}} \))}

\[
\text{GDP}_{\text{pc,new}} = \frac{\text{GDP}_{\text{new}}}{P}
\]

Substituting \( \text{GDP}_{\text{new}} \):
\[
\text{GDP}_{\text{pc,new}} = \frac{\text{GDP}_{\text{current}} + (\beta_C + \beta_I + \beta_G) \cdot (\Delta T + \Delta \text{Tariff}) \cdot \text{GDP}_{\text{current}}}{P}
\]

\subsection*{Change in Wealth Disparity (\( \Delta G \))}

\[
\Delta G = (\alpha_Y + \alpha_RR) \cdot (\Delta T + \Delta \text{Tariff}) \cdot G_{\text{current}}
\]

\subsection*{New Wealth Disparity (\( G_{\text{new}} \))}

\[
G_{\text{new}} = G_{\text{current}} + \Delta G
\]

\section{Results}

The data gathered for this study had to utilize multiple sources. This is because we do not have access to the exact GDP spending, however we were able to utilize other sites that have reported on this data to gather a rough estimate. In order to be accurate since the year 2024 has not yet concluded we decided to implement data from 2023 since it is the most recent completed fiscal year. By doing this we were better able to see the total impact of the policy actions Donald Trump Plans to implement.

\subsection{Parameters}

\subsection*{Economic Data (2023)}

\begin{itemize}
    \item Current GDP: \( \text{GDP}_{\text{current}} = 27.36 \, \text{trillion USD} \) \citep{BEA}
    \item Population: 334.9 million \citep{WorldM}
    \item Percentage of wealth held by the top 10\%: \( G_{\text{current}} = 60\% \) \citep{Fiscal}
    \item GDP Per Capita: \( \text{GDP}_{\text{pc,current}} = 81,695 \, \text{USD/person} \) \citep{Macro}
\end{itemize}

\subsection*{Shares of GDP (2023)}

\begin{itemize}
    \item Government spending as a share of GDP: \( \beta_G =0.207 \, \text{(20.7\%)} \)\citep{Fiscal}
    \item Personal consumption expenditures as a share of GDP: \( \beta_C =0.6817 \, \text{(68.17\%)} \) \citep{BEA}
    \item Investment as a share of GDP: \( \beta_I = 0.207 \, \text{(20.7\%)} \) \citep{CEIC}
\end{itemize}

\subsection*{Policy Changes}

\begin{itemize}
    \item Average income tax rate: \( \Delta T = 0.22 \, \text{(22\%)} \) \citep{IRS}
    \item Tariff increase: \( \Delta \text{Tariff} = 0.75 \, \text{(from 22\% to 100\%)} \) \citep{TaxFound}
    \newline This is how high he desires but the actual policy will most likely 
            be lower. We evaluated for 100\% since that is Trump's desired outcome.
\end{itemize}

\subsection*{Elasticity Parameters}

\begin{itemize}
    \item GDP elasticity with respect to taxes and tariffs:
    \[
    \beta_C + \beta_I + \beta_G = 0.6817 + 0.207 + 0.207 = 1.0957
    \]
    \item Wealth disparity elasticity with respect to taxes and tariffs:
    \[
    \alpha_Y + \alpha_RR = 0.4 + 0.1 = 0.5
    \]
\end{itemize}

\subsection{Calculations}

\subsection*{Change in GDP (\( \Delta \text{GDP} \))}

\[
\Delta \text{GDP} = (0.6817 + 0.207 + 0.207) \cdot (0.22 + 0.75) \cdot 27.36
\]
\[
\Delta \text{GDP} = 1.0957 \cdot 0.97 \cdot 27.36 = 29.12 \, \text{trillion USD}
\]

\subsection*{New GDP (\( \text{GDP}_{\text{new}} \))}

\[
\text{GDP}_{\text{new}} = 27.36 + 29.12 = 56.48 \, \text{trillion USD}
\]
\begin{figure} [ht]
    \centering
    \includegraphics[width=1.1\linewidth]{Change in GDP.png}
    \caption{New GDP}
    \label{fig:1}
\end{figure}

\indent As the tax and tariff rate increases, the total GDP grows substantially. This is modeled by the elasticity of the GDP with respect to the tax and tariff policy. The increase in GDP suggests that higher tariffs and tax revenues could boost economic activity or government spending in this model.

\subsection*{New GDP Per Capita (\( \text{GDP}_{\text{pc,new}} \))}
\[
\text{GDP}_{\text{pc,new}} = \frac{\text{GDP}_{\text{new}}}{P}
\]
\[
\text{GDP}_{\text{pc,new}} = \frac{56.48}{0.3349} = 168.64 \, \text{thousand USD/person}
\]

\begin{figure} [ht]
    \centering
    \includegraphics[width=1.1\linewidth]{Change in GDP Per Capita.png}
    \caption{Change in GDP Per Capita}
    \label{fig:2}
\end{figure}

\indent There is a steady increase in GDP per capita. The initial GDP Per capita as of 2023 was \$81,695 and increased to approximately \$128,720 as the policy reached maximum output. This assumes that the population has remained constant and that economic gains are distributed proportionately. 

\subsection*{Change in Wealth Disparity (\( \Delta G \))}
\[
\Delta G = (0.4 + 0.1) \cdot (0.22 + 0.75) \cdot 60
\]
\[
\Delta G = 0.5 \cdot 0.97 \cdot 60 = 29.1\%
\]

\subsection*{New Wealth Disparity (\( G_{\text{new}} \))}

\[
G_{\text{new}} = G_{\text{current}} + \Delta G
\]
\[
G_{\text{new}} = 60 + 29.1 = 89.1\% \, \text{(Wealth held by top 10\% increases to 89.1\%)}
\]
\begin{figure}[ht]
    \centering
    \includegraphics[width=1.1\linewidth]{Change in Wealth Disparity.png}
    \caption{Change in Wealth Disparity}
    \label{fig:3}
\end{figure}

\indent The graph indicates that wealth disparity increases. The percentage of wealth held by the top 10\% of earners starts at 60\% and grows to 89\%. This outcome assumes that higher taxes and tariffs disproportionately impact lower-income groups and that wealthier groups benefit more from the economic gains.

\section{Conclusion}

\indent Confounders in our research such as us not having access to specific metrics should be acknowledged to determine if this is a realistic result. Assuming our data is accurate we can state that the models do suggest that high taxes and tariffs could drive significant GDP and GDP per capita growth. The correlated increase in wealth disparity is very concerning however since it implies the growth in GDP may not be evenly distributed. Overall one must analyze on their own if the growth in GDP and GDP per capita is worth the trade-off of a growth in inequality and possibly a higher poverty rate. Since our data isn't completely sound we can't provide 100\% accurate policy solutions, however assuming it is accurate we recommend possible acts that will assure for equal distribution of the GDP growth to ease the spike in wealth disparity. These acts include higher property tax for the uber wealthy, as well as a redistribution of earnings into welfare programs to help those outside of the top 10 percent. For future research we recommend recreating our study with the more accurate data provided by the government as well as analyzing how individual states may be impacted.


\newpage
\bibliographystyle{apalike}
\bibliography{Citations}
\citep{Max_Zahn}
\citep{Fiscal}
\citep{BEA}
\citep{CEIC}
\citep{IRS}
\citep{TaxFound}
\citep{Macro}
\citep{WorldM}
\citep{Vermeer_Timothy}
\citep{Hellmann_Melissa}
\citep{Steindel_Charles}
\citep{Cornforth_Ed}
\citep{Goss_Jacob}
\citep{Kogan_Bobby}
\end{document}
